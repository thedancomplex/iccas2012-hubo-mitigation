\section{Methodology}

The robot that will be used is the adult-size humanoid robot Jaemi Hubo, see Fig.~\ref{fig:huboSch}.  Jaemi Hubo is a high gain position controlled device.  Each actuator is fed a reference position and is able to feed back the actual position, the current through the actuator, and the actuator status (enabled or disabled).  Each actuator has an over current automatic disable routine.  When this routine is executed the motor control is deactivated but feedback of the actual position is still enabled.  This routine was determined to be the cause of Jaemi Hubo's ankle failure.

\begin{figure}[thpb]
  \centering
\includegraphics[width=1.0\columnwidth]{./pix/huboSch.png}
  \caption{Jaemi Hubo: 130cm tall 45kg (with battery and protective shell) 40 degree of freedom, high gain position controlled adult-size humanoid robot }
  \label{fig:huboSch}
\end{figure}

Real time constraints. Difficult to simulate faults with robot.

Collect data from all metrics, and construct convex hull for normal state. If metric is not useful, it will not affect hull (metric is independent of other metrics).

Faults include over torque of actuators, loss of zero-moment-point, over-current disable

Mitigations include: restart actuator, reduce current (disable noncritical actuator?), rebalance (shift position in various ways, move body, make sure feet are on ground, etc.)