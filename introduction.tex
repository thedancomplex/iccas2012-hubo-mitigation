\section{Introduction}
\label{sec:introduction}
\textit{April 27th, 2010 - Philadelphia Convention Center (Main Hall):} The
robot handlers Daniel M. Lofaro and Robert Ellenberg were preparing the
adult-size humanoid robot Jaemi Hubo\footnote{Jaemi Hubo Home Page:
	http://dasl.mem.drexel.edu/HUBO} at an outreach demonstration for the
\textit{Arts \& Science Council} of Philadelphia.  During the dress rehearsal
one of Jaemi's actuators failed while preforming an active balancing demonstration.  This resulted in
Jaemi Hubo falling off the 4 foot high stage.  The results of the impact can be found in
Fig.~\ref{fig:fall} and on YouTube\footnote{Jaemi Hubo Fall:
	http://www.youtube.com/watch?v=DF8zAM4FLB4}\label{link:fall}.  This annual
event was a high profile fund raiser for the arts and science programs
throughout the greater Philadelphia area and was covered widely by the media.
If this failure would have occurred during the actual event the aftermath would
have been even more devastating.  Jaemi Hubo was repaired in full by Lofaro, Ellenberg 
and the rest of the Drexel Autonomous Systems Lab (DASL) within 50 days of the accident, see Fig.~\ref{fig:huboFix}.  However 
in order for outreach events like this to
successfully continue, methods for detecting failure states and quickly
choosing appropriate mitigations must be developed.

%This work shows the authors focuses on outlining the how to detect and
%classify failure states (Section~\ref{sub:FailureStateDetermination}),

All electro-mechanical systems have an inherent mean-time to failure.  Even
with good maintenance these systems can fail without warning.  This work
proposes ways to \textit{detect} when entering a failure state and ways of
\textit{mitigating} such failures.  A failure state is defined as the operating
conditions where the robot is unable to safely preform tasks.  This includes, but is
not limited to, actuator faults and failures, loss of balance, power loss, etc.
Faults are defined as the failure (intermittent or perpetual) in a single part
of the robot.  A single fault does not mean a system failure in all cases.  The
adult-size humanoid robots Hubo2+ and Jaemi Hubo (KHR-4) are the primary test
platforms for this proposed work.  All methods used are written in a broad
scope so it can be applicable to other electro-mechanical systems.

\begin{figure}[t]
  \centering
\includegraphics[width=1.0\columnwidth]{./pix/jaemiFall.png}
  \caption{- Aftermath of the 4 foot fall Jaemi Hubo took after one of her
		actuators failed during operation.  A video with more images of the
		aftermath of the failure and further explanation of the event can be seen
		on YouTube$^\ref{link:fall}$}
  \label{fig:fall}
\end{figure}

Faults are difficult to detect before an executing system reaches a point of
failure, as the first symptom of a fault is often system failure itself. While
it is unrealistic to expect complex systems to be fault-free, actions such as
resetting the system, quarantining specific components, or minimizing damage
from the fault can be taken. Autonomic systems, an extension of fault tolerant
systems, attempt to detect, diagnose, and mitigate faults quickly. These
systems are inspired by the autonomic nervous system in the human body that
monitors and regulates vital functions of the body such as heart rate,
respiration rate, and digestion. Similarly, an autonomic computer system is
able to monitor itself and its environment and automatically adapt to complex
changes. The goal of autonomic computing is to specify the desired state of a
system using high-level objectives without detailing how to arrive at the state
\cite{1160055,4061119,1301340}. By making intelligent decisions, autonomic
systems free system administrators from low-level management and the
intricacies of complex systems. Autonomic systems aim to be self-configuring,
self-optimizing, self-healing, and self-protecting. These properties,
collectively referred to as the self-* properties, are different views of the
same self-man\-age\-ment property. For instance, a self-protecting system is
ideally healing itself from faults while optimizing and reconfiguring itself to
prevent other faults from reoccurring.

%Outlined in this document is the proposed process to analyze the effects of
%faults on the complex system Jaemi Hubo/Hubo2+ (Fig.~\ref{fig:huboSch}).  We
%propose a method of injecting faults in a controlled environment so that we
%may easily analyze the effects of the faults on the normal operating state,
%Section~\ref{sub:FailureStateDetermination}~.  Different mitigation techniques
%and their effect on the operating state will be explored and analyzed,
%Section~\ref{sub:mitigationanalysis}~.

In this work Section~\ref{sub:FailureStateDetermination} outlines the authors
plan to use their bleeding-edge software engineering failure state detection
techniques on the complex electro-mechanical system Jaemi Hubo.
Section~\ref{sec:faultInjection} describes the system faults used to determine failure
states and how these faults are injected into the system in a controlled environment.  Finally
Section~\ref{sub:mitigationanalysis} shows how to define proper mitigation
techniques for a give failure state and Section~\ref{sec:expResults} describes
our expected results.

\begin{figure}[thpb]
  \centering
\includegraphics[width=0.9\columnwidth]{./pix/jaemiFix3.png}
  \caption{Jaemi Hubo 50 days after the fall at the Philadelphia Convention Center.  
  Jaemi Hubo is once again in full operational order.  She was fixed solely by the students at 
  the Drexel Autonomous Systems Lab (DASL).  This demonstrates the successful transfer of tribal knowledge 
  of the Hubo platform from the Hubo Lab at KAIST to the Drexel Autonomous Systems Lab at Drexel University.
	}
  \label{fig:huboFix}
\end{figure}