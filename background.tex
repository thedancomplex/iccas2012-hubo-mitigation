\section{Background}
\label{sec:background}
The techniques proposed in this work borrows from the bleeding edge work in
autonomic software engineering and computing.  The heart of autonomic computing
is anomaly detection, diagnosis, and mitigation. Autonomic systems perform 4
general tasks in a continuous closed loop: monitor components with the help of
sensors, interpret the monitored data, create a repair plan for system
adaptation, and execute this plan through effectors on the monitored system and
its environment. This is the MAPE (Monitor, Analyze, Plan, Execute) loop
\cite{1160055}. Fig.~\ref{fig:mape} illustrates how an autonomic element uses
this loop to manage a component.

\begin{figure}[tb]
  \centering
  \includegraphics[width=\columnwidth]{images/mape}
  \caption{- An autonomic element uses the MAPE control loop to monitor a
		component, analyze the its status with respect to a policy, plan how to
		meet the policy requirements, and execute a set of actions to do so.}
	\label{fig:mape}
\end{figure}

Autonomic approaches can be applied to many different types of systems,
particularly when focusing on fault detection and mitigation. self-healing
opearting systems \cite{DBLP:journals/ibmsj/AppavooHSWSKAEGGMORSX03,4351383}
protect against bit-flips and other transient hardware faults, and are able to
hot-swap components or firmware during runtime. At the application level, it is
possible to monitor the architecture requirements of a system and detect when
an architectural property (e.g. required connection bandwidth) falls outside of
an acceptable threshold \cite{Garlan2004}. If an invariant is violated, it
executes a strategy specifically designed to correct the violated
invariant. The strategies can reinitialize components and reconfigure the
system to recover from the violation and attempt to prevent the violation from
reoccurring. However, the applied strategies are \emph{ad hoc} solutions,
requiring developers and maintainers to understand and preempt the shortcomings
of the system, in much the same way they would manually debug and repair the
system.

\subsection{Fault Detection and Diagnosis}
\label{sub:fault_detection}
Previous approaches to the detection of software faults fall into two
categories, signature-based and
anomaly-based~\cite{al-nashif2008}. Signature-based methods detect faults by
matching measurements to known fault signatures. These techniques are used in
static fault-checking software such as the commercial antivirus software
McAfee~\cite{Mcafee} and Symantec~\cite{symantec}, as well as network intrusion
detection systems such as Snort~\cite{Roesch1999} and
Netstat~\cite{Vigna1999}. These techniques can also be used to detect recurring
runtime faults~\cite{Brodie2005}.

Typically, anomaly-detection techniques begin by collecting sensor measurements
of a normally behaving system. Then, they construct a representation of the
monitored system and compare any future measurements against that
representation. A common approach is to use metric correlations to quantify a
monitored system. During detection, if the correlations between metrics becomes
significantly different from the learned correlations, the system is classified
to be in a faulty state~\cite{zhen2006,zhao2009,Cohen2004,Jiang2009}.

Once a fault is detected, it must be correctly diagnosed. Bayesian Network
classifiers have been applied in several previous approaches to diagnosis
using. Ghanbari et al. propose an approach to anomaly diagnosis based on
Bayesian networks that are wholly or partially specified by a human user
\cite{Ghanbari2008}. Tree augmented naive bayesian classifiers are the basis
for software failure diagnosis in work by Cohen et al.~\cite{Cohen2004}. Zhang
et al. use ensembles of tree augmented naive Bayesian classifiers to diagnose
faults.~\cite{Zhang2005}. Other classification techniques have been used to
diagnose software faults. Chen et al.  offer an approach to the diagnosis of
failures in large-scale Internet service systems using decision trees
\cite{Chen2004}.  Duan et al.~\cite{Duan} propose an approach to diagnosis of
software failures that uses active learning to minimize the number of data
points that a human administrator must label.

In previous work it has been demonstrated that computational geometry can be
used to effectively detect system faults at
runtime\cite{stehle2010,shevertalov2010using}. The approach, introduced in the
work as Aniketos, is split into a training phase, and a monitoring
phase. During the training phase, the monitored system executes its normal
behavior and runtime data is periodically collected from sensors monitoring the
application, the operating system, and the hardware. Using these measurements,
we can construct an $n$-dimensional convex hull whose enclosing space
represents the normal execution of the monitored application, where $n$ is the
number of distinct metrics used. Figure~\ref{fig:class_hulls} shows an example
of a 2-dimensional convex hull.

\begin{figure}%[tb]
  \centering
  \includegraphics[width=\columnwidth]{images/class_hulls}
  \caption{- Building convex hulls for the diagnosis of faults (a) sample
		points from normal operation and two fault classes (b) Points inside of the
		convex hulls are diagnosed as normal or one of the fault classes. Points
		outside of all hulls are diagnosed as unknown faults.}
  \label{fig:class_hulls}
\end{figure}

During the monitoring phase, if a measurement point falls outside of this
enclosure, a fault is likely to have occurred. To properly mitigate the fault,
the problem must be diagnosed \cite{6100108} and a viable mitigation
selected. If measurement data is available for a set of known faults, then it
is possible to construct hulls for individual faults, or classes of
faults. During the monitoring phase, if a measurement falls within one of these
fault hulls we can determine which fault is likely occurring and apply a
mitigation.This approach is also capable of recognizing that an occurring fault
is not a type of fault that it has been trained to recognize. Typical
classification techniques, such as naive Bayes and voting feature
intervals\cite{Demiroz1997}, require that every sample be assigned a class from
a finite list of possible classes, which can cause drastic consequences if an
incorrect mitigation is selected. Our approach allows Aniketos to recognize
that a fault has occurred, but does not force Aniketos to label the fault as
one of the faults that it has been trained to recognize. One of the most
important features of a fault detection and diagnosis system is the ability to
handle new faults.

\begin{figure}[thpb]
  \centering
\includegraphics[width=1.0\columnwidth]{./pix/huboSch.png}
  \caption{Jaemi Hubo: 130cm tall 45kg (with battery and protective shell) 40
		degree of freedom, high gain position controlled adult-size humanoid robot
	}
  \label{fig:huboSch}
\end{figure}

\subsection{Fault Mitigation}
\label{sub:fault_mitigation}
Once a fault is detected and diagnosed a mitigation can be applied in an
attempt to bring the system back to a normal operating state. To select the
most appropriate mitigation to execute, mitigations must be evaluated based on
their performance cost, impact on the system, reliability, and how they affect
other mitigations locally as well as in the global environment. Given a set of
faults and a set of mitigations, it is possible to intuitively and empiracally
determine the best possible mitigation for a given fault. However, it is
impossible to plan for all possible fault scenarios, limiting the effectiveness
of mitigation selection.

Generally, when a fault is encountered there are several approaches that can be
taken. Reinitialization brings the whole system or part of the system, to its
initial state when a fault is detected.If a fault can be localized, it is
possible to progressively microreboot larger portions of the system until the
problem is resolved, starting with the most localized component
\cite{1251257}. A generalization of reinitialization, known as rollback,
restores the system to a previous checkpoint, undoing any environment changes
that may have been made \cite{1095833}. While this approach may not always be
feasible, in some cases, simply undoing the failing operation may be
sufficient. Often times, the system needs to be reconfigured to adapt to
changes in the system's environment. This can be done by tuning component
parameters, scaling components, replacing components, or removing components
entirely. Finally, in certain cases, it may be best to just accept the fault
and continue without mitigation.

It is important to understand that each of these approaches could have
potentially disastrous consequences on a given system. Different systems have
different objectives and costs risks associated with it. Similarly, each
mitigation has execution costs and risks associated with it. It is possible for
a selected mitigation to cause more faults, or increased downtime. If an
incorrect mitigation is applied, a cascading effect could quickly emerge.


\subsection{Humanoid Robot Fault Mitigation}
\label{sub:humanoidRobotFaultMitigation}
A typical fault for articulated robots is actuator failure.  Upon failure there
is no more power applied to the joint and the entire limb becomes under-actuated.
This is particularly hazardous to robots that require feedback to balance such as 
biped humanoid robots.  For biped humanoids these errors typically caused actuator 
over torque or hardware error resulting in loss of zero-moment-point
(ZMP) \cite{zmp35} causing a robot fall or collapse.  This is exceptionally
harmful to adult size humanoid robots due to their weight.  A common 
mitigation method described by Shin, J. et al.\cite{772398} from the Korean
Advanced Institute of Science and Technology (KAIST) changes the model of the
affected manipulator to one that is under-actuated.  This new model allows
the robust controller to continue to operate collision free (safe).

Current methods of mitigation of ZMP loss for biped humanoids been investigated
by Kiyoshi Fujiwara et al. \cite{4115653}.  These methods involve finding an
optimal falling trajectory that reduces the instantaneous force of the robot at
impact by creating multiple impact stages\cite{4399327}.  This method was fully
tested on an HRP-2FX (HRP-2P surrogate) and partially on an HRP-2P.  This work
did not include a method of determining a falling state, it is assumed that a
fall is in progress.  Additional work on detecting a fall and reducing fall
damage has been shown by Kunihiro Ogata et al.\cite{4755950}.  An active
shock-reducing motion reduces the impact damage by following the center of
gravity (COG) and attempting to keep it close to the ZMP support polygon.  The
falling state is determined when the predicted ZMP departs from the support
polygon. This method was tested on a miniature humanoid robot.  Additional work
on determining a fall state using machine learning techniques\cite{4813885}.
Reimund Renner et al. used parameter estimation of multiple sensors to detect a
falling state\cite{4058847}.
