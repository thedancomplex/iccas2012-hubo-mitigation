\section{Introduction}
Electro-mechanical systems eneviatiaibly fail during use.  Failors during opperation typically cause system haults.  This is particuralaly hazerdous to robots that require feedback to ballance such as biped humanoid robots.  For biped humanoids these errors typically include, but are not limited to, actuator failure due to over torque or loss of zero-moment-point (ZMP) \cite{zmp35} causing a robot fall or colapse.  This is exceptionally harmful to adult size humanoid robots due to their weight.  This paper focuses on way of automatically detecting when a failure is about to occure and choosing the proper mitigation technique. 

Current methods of mitigation of ZMP loss for biped humanoids been invistigated by Kiyoshi Fujiwara et al. \cite{4115653}.  These methods involve finding an optiomal falling trajectory that reduces the instanous force of the robot at impact by creating multiple impact stages\cite{4399327}.  This method was fully tested on an HRP-2FX (HRP-2P surragate) and partiailly on an HRP-2P.  This work did not include a method of determining a falling state, it is assummed that a fall is in progress.  Additional work on detecting a fall and reducing fall damage has been shown by Kunihiro Ogata et al.\cite{4755950}.  An active shok-reducing motion reduces the impact damage by following the center of gravity (COG) and attempting to keep it close to the ZMP support polygon.  The falling state is determined when the predicted ZMP departs from the support polygon. This method was tested on a miniture humanoid robot.  Addiotonal work on determining a fall state using machine learning techniques\cite{4813885}.  Reimund Renner et al. used paramater estimation of multiple sensors to detect a falling state\cite{4058847}.

These methods are able to detect a spisific fault state, however a need of detecting different fault states requires more general methods.  These methods must include monitoring not only sensor data but also actuator failor status.  