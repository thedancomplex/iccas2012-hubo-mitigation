
Electro-mechanical systems inevitably fail during use.  Failures during operation typically cause system halts.  This is particularly hazardous to robots that require feedback to balance such as biped humanoid robots.  For biped humanoids these errors typically include, but are not limited to, actuator failure due to over torque or loss of zero-moment-point (ZMP) \cite{zmp35} causing a robot fall or collapse.  This is exceptionally harmful to adult size humanoid robots due to their weight.  It is key for the robot to recognize when it is entering a failure state and be able move to a safe running state.  This paper focuses on way of automatically detecting when a failure is about to occur and choosing the proper mitigation technique.

 

Current methods of mitigation of ZMP loss for biped humanoids been investigated by Kiyoshi Fujiwara et al. \cite{4115653}.  These methods involve finding an optimal falling trajectory that reduces the instantaneous force of the robot at impact by creating multiple impact stages\cite{4399327}.  This method was fully tested on an HRP-2FX (HRP-2P surrogate) and partially on an HRP-2P.  This work did not include a method of determining a falling state, it is assumed that a fall is in progress.  Additional work on detecting a fall and reducing fall damage has been shown by Kunihiro Ogata et al.\cite{4755950}.  An active shock-reducing motion reduces the impact damage by following the center of gravity (COG) and attempting to keep it close to the ZMP support polygon.  The falling state is determined when the predicted ZMP departs from the support polygon. This method was tested on a miniature humanoid robot.  Additional work on determining a fall state using machine learning techniques\cite{4813885}.  Reimund Renner et al. used parameter estimation of multiple sensors to detect a falling state\cite{4058847}.

These methods are able to detect a specific fault state, however a need of detecting different fault states requires more general methods.  These methods must include monitoring not only sensor data but also actuator failure status.